\section{Related Work}
\label{sec:rel}

Er is al eerder onderzoek gedaan naar het verschil in gender op Wikipedia. Door middel van vragenlijsten is onderzocht naar het verschil \todo{dit is geen nederlands} in aantal mannelijke en vrouwelijke Wikipedia editors. Hieruit kwam dat ongeveer 16,1\% van de editors op de online encyclopedie van het vrouwelijke geslacht is \citep{hill2013wikipedia}. Binnen Amerika lag dit wel een beetje hoger, namelijk 22,7\% van de editors was hier van het vrouwelijke geslacht. Hoewel Wikipedia meer artikelen heeft over vrouwen dan andere online encyclopedia zoals Britannica, is het zo dat binnen Wikipedia zelf relatief weinig artikelen gaan over vrouwen ten opzichten van artikelen over mannen \citep{reagle2011gender}. Hieruit lijkt het redelijk duidelijk dat vrouwen worden ondervertegenwoordigd. Het is daarom belangrijk om af te vragen waardoor dit komt. Ligt het bijvoorbeeld aan persoonlijke voorkeur, aantrekkelijkheid of discriminatie. Door deze oorzaak te begrijpen kan worden ingegrepen om balans te herstellen. 



\todo{Ik vind de twee onderstaande stukjes heel goed, maar die horen in de introductie. Je motiveert hierin jouw onderzoeksvragen heel goed, en haalt er literatuur bij over wikipedia zelf, en aanvullende meer psychologisch literatuur (dat denk ik althns, de links werken niet, en staan niet in je bib file). \\
Verplaats dit dus naar je intro, en laat het zo. 
\\
In je related work hoort ander onderzoek gedaan op de edit-history/discussie paginas van Wikipedia. Geef daar een volledig overzicht van. Zoveel is het toch ook weer niet. En bespreek dan steeds kort wat ze wilden weten en wat voor technieken ze gebruikten. Helemaal als jij iets vergelijkbaars doet.}

\subsection{Kritiek}
Hoewel Wikipedia bekent staat dat alle mensen iets kunnen 'bijdragen' aan de gemeenschap, is het vaak zo dat 'bijdragen' staat voor het verwijderen of aanpassen van andermans werk. 'Samenwerken' betekent dus niet altijd dat iedereen collectief nieuwe informatie produceert. Vaak is 'samenwerken' het feit dat mensen het werk van onbekende anderen lezen, corrigeren en soms ook verwijderen. \todo{Mooi, maar graag referenties hierbij, het gaat tenslotte om related work} Er is al eerder onderzocht dat als kinderen, jongens sneller kiezen voor competitieve spelletjes, terwijl meisjes sneller kiezen voor spelletjes zonder een winnaar \citep{campbell2013mind}. Een reden hiervan is dat mannen meer houden van competitie. Vrouwen reageren ook anders op kritiek dan mannen. Mannen reageren minder op positieve of negatieve feedback dan vrouwen \citep{roberts1994gender}. Vrouwen krijgen meer zelfvertrouwen van positieve feedback terwijl mannen relatief onaangetast blijven. Bij het ontvangen van negatieve kritiek wordt het zelfvertrouwen van vrouwen aanzienlijk verlaagd terwijl dit van mannen wederom onaangetast blijft \citep{roberts1994gender}. Vrouwen gaven ook aan 31\% banger te zijn voor negatieve houdingen ten opzichten van hun werk op Wikipedia \citep{collier2012conflict}. Een reden voor het lage aantal vrouwelijke editors op Wikipedia zou dus kunnen zijn dat mannen negatiever zijn dan vrouwen. 

\todo{maar hier hoort je onderzoeksvraag toch niet?}
\begin{description}
\item[Onderzoeksvraag 1] Is het taalgebruik van mannelijke gebruikers negatiever dan het taalgebruik van vrouwelijke gebruikers op discussiepagina's van Wikipedia?
\end{description}

\subsection{Conflict}
Behalve de kritieke sfeer tussen Wikipedia editors, is er ook een hoog niveau aan conflicten. Hoewel het lijkt dat er veel wordt samengewerkt op Wikipedia, staan de discussiepagina's vol met \textit{edit-warring}: een fenomeen waar editors proberen andere edits op te heffen \citep{cassell2011editwars}. Volgens Cassell moeten editors erop aandringen dan hun standpunt de enige geldige is. Eerder onderzoek heeft al aangetoond dat vrouwen sneller geneigd zijn om conflicten te vermijden dan man \citep{brewer2002gender}. Ook ervaren vrouwen sneller angst en verhoogde hartslag bij conflicterende situaties \citep{smith1998agency}. Eerder onderzoek toonde ook aan dat vrouwen onderhandelingen en conflict zelfs vermijden als het gaat om hoge kosten. Een vrouw is bereid ongeveer \$1300,- meer te betalen voor een auto om onderhandelingen te vermijden \citep{babcock2009women}. Omdat Wikipedia's discussiepagina's grotendeels zijn gebaseerd op het oplossen van conflicten, persoonlijke discussies en onderhandelingen om tot een consensus te komen, zouden de verschillen in gender om mee te doen aan deze activiteiten de impact van contributie kunnen be\"{i}nvloeden.
\begin{description}
\item[Onderzoeksvraag 2] Zijn er vaak discussies tussen mannelijke gebruikers en vrouwelijke gebruikers op discussiepagina's?
\end{description}
