\section{Related Work}
\label{sec:rel}

Er is al eerder onderzoek gedaan naar het verschil in gender op Wikipedia. Door middel van vragenlijsten is onderzocht naar het verschil tussen mannelijke en vrouwelijke Wikipedia editors. Hieruit kwam dat ongeveer 16,1\% van de editors op de online encyclopedie van het vrouwelijke geslacht is \citep{hill2013wikipedia}. Binnen Amerika lag dit wel een beetje hoger, namelijk 22,7\% van de editors was hier van het vrouwelijke geslacht. Hoewel Wikipedia meer artikelen heeft over vrouwen dan andere online encyclopedia zoals Britannica, is het zo dat binnen Wikipedia zelf relatief weinig artikelen gaan over vrouwen ten opzichten van artikelen over mannen \citep{reagle2011gender}. Hieruit lijkt het redelijk duidelijk dat vrouwen worden ondervertegenwoordigd. Het is daarom belangrijk om af te vragen waardoor dit komt. Ligt het bijvoorbeeld aan persoonlijke voorkeur, aantrekkelijkheid of discriminatie. Door deze oorzaak te begrijpen kan worden ingegrepen om balans te herstellen.

\citet{viegas2007talk} onderzochten 25 willekeurige discussiepagina's uit een dump uit 2005 van de Engelstalige Wikipedia. Zij cre\"{e}erden 11 categori\"{e}n waarin gebruikers die dingen op discussiepagina's plaatsten ingedeeld konden worden. De meest voorkomende categorie was \textit{request for coordination} binnen de discussiepagina's. Dit is welliswaar ook de functie van de desbetreffende pagina's. Op de tweede plaats stond \textit{request for information}. Dit zijn gebruikers die geen intentie hebben om de artikelen aan te passen, maar kennis willen opdoen van andere mensen via de discussiepagina's. \citet{viegas2007talk} ondervonden dat converstaties op discussiepagina's redelijk geformaliseerd zijn en speciale etiketten aanhouden.

\citet{laniado2011wikipedians} deden een uitgebreide analyse van de discussiepagina's van artikelen en van gebruikers. De focus lag op het detecteren van structurele patronen van interactie op de pagina's. Door te kijken naar de reacties die gebruikers gaven op discussiepagina's van artikelen of van andere gebruikers werd een een netwerk opgebouwd. Zij ondervonden dat gebruikers die veel \textit{outgoing links} hebben vaak reageren op gebruikers met weinig outgoing links en vice versa. Ervaren gebruikers reageren dus vaak op onervaren/nieuwe gebruikers. Ook werd er gekeken naar de grootte en vorm van de discussies op de discussiepagina's door te kijken naar het netwerk dat ontstond. Een voorbeeld is dat bij wiskunde er heel uitgebreid werd gediscussieerd maar er weinig verschillende gebruikers discussieerden en er relatief minder edits werden gedaan aan het corresponderende artikel dan men zou denken.

\citet{kittur2007he} gebruikten berekenbare metriek en standaard \textit{machine learning} technieken om zo succesvol het niveau van conflict in een artikel te voorspellen. Op pagina's waar wellicht een heftige discussie plaats zou kunnen vinden staat een marker "{{controversial}}". Eerst werd de Controversial Revision Count (CRC) berekent door te kijken of deze marker op een pagina stond. Op de pagina's met een score van hoger dan 0 werden de metrieken vergeleken. Dit waren:

\begin{itemize}
\item Aantal revisies (discussiepagina) $\uparrow$
\item Aantal kleine edits (discussiepagina) $\uparrow$
\item Aantal unieke gebruikers (discusiepagina) $\downarrow$
\item Aantal revisies (artikelpagina) $\uparrow$
\item Aantal unieke gebruiekrs (artikelpagina) $\downarrow$
\item Aantal anonieme edits (discussiepagina) $\uparrow$
\item Aantal anonieme edits (artikelpagina) $\downarrow$
\end{itemize}

De pijlen erachter geven aan of het een positieve ($\uparrow$) of negatieve ($\downarrow$) correlatie heeft met mate van conflict. Met deze \textit{training set} konden \citet{kittur2007he} succesvol conflicten voorspellen op Wikipediapagina's, zelfs als ze geen \textit{controversial} marker hadden.



