\section{Introduction}
\label{sec:intro}

Wikipedia is een voorbeeld van de grootste online community van collaborerende mensen op het internet. Op Wikipedia zijn er veel users die edits maken op pagina's. Zij maken nieuwe pagina's, bewerken pagina's of controleren bewerkingen van andere users. Achter elke pagina schuilt ook een discussiepagina. Deze discussiepagina's dienen voor coordinatie, discussies en communicaties tussen \textit{Wikipedians}. In tegenstelling tot andere veel andere websites, zijn discussiepagina's op Wikipedia dus niet bedoelt voor entertainment, maar om tot een consensus te komen wat zal zorgen tot verbetering van Wikipedia pagina's \citep{laniado2011wikipedians}.  Echter, uit een vragenlijst bleek dat minder dan 20\% van deze Wikipedians vrouw is \citep{glott2010analysis}. Dit grote verschil tussen aantal mannen en aantal vrouwen binnen deze Wikipedians heeft veel aandacht gekregen van zowel de media als andere onderzoekers. Gender onderzoeksliteratuur \citep{collier2012conflict} suggereert dat dit verschil kan komen door:
\begin{enumerate}
    \item Het hoge aantal conflicten in discussies
    \item Afkeer van kritische omgevingen
    \item Te kort aan zelfvertrouwen bij het aanpassen van andermans werk.
\end{enumerate}  
Er zijn al een aantal onderzoeken gedaan naar het analyseren van de inhoud van Wikipedia discussiepagina's \citep{viegas2007talk} en de structuur van de discussiepagina's van Wikipedia \citep{laniado2011wikipedians}, maar nog niet naar het verschil tussen mannenlijke en vrouwlijke verschillen op discussiepagina's van Wikipedia. In deze thesis wordt er gekeken naar het verschil in gebruik van de discussiepagina's van Wikipedia tussen mannelijke en vrouwlijke Wikipedians.



\subsection{Onderzoeksvragen}
Om verschil tussen mannen en vrouwen op Wikipedia talkpages beter te bekijken, zijn er een aantal onderzoeksvagen opgesteld. 

\begin{description}
\item[Onderzoeksvraag 1] Zijn er vaak discussies tussen mannelijke gebruikers en vrouwelijke gebruikers op discussiepagina's?

\item[Onderzoeksvraag 2] Is het taalgebruik van mannelijke gebruikers negatiever dan het taalgebruik van vrouwelijke gebruikers op discussiepagina's van Wikipedia?

\subsection{Inhoud van thesis}


\begin{comment}
\begin{itemize}
\item Bevat je onderzoeksvraag (of vragen)
\item Plaatst je vraag in de bestaande literatuur.
\end{itemize}

Je onderzoeksvraag is leidend voor je hele scriptie. Alles wat je doet moet uiteindelijk terug te voeren zijn op 1 doel: het beantwoorden van die vraag. 

Typisch zal je het dan ook zo doen:

Mijn onderzoeksvraag is onderverdeeld in de volgende deelvragen:

\begin{description}
\item[RQ1] \ldots We   beantwoorden deze vraag  door het volgende te doen/ antwoord op de volgende vragen te vinden/ \ldots
\begin{enumerate}
\item Vragen op dit niveau kan je echt beantwoorden, en dat doe je in je Evaluatie sectie~\ref{sec:eva}.
\end{enumerate}
\item[RQ2] \ldots
\item[RQ3] \ldots
\end{description}

Je Evaluatie sectie~\ref{sec:eva} bevat evenveel subsecties als je deelvragen hebt. En in elke sectie beantwoord je dan die deelvraag met behulp van de vragen op het onderste niveau.

In je conclusies kan je dan je hoofdvraag gaan beantwoorden op basis van al het eerder vergaarde bewijs.


\paragraph{Overview of thesis}
Hier geef je even kort weer wat in elke sectie staat.

\end{comment}

