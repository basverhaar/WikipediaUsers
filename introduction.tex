\section{Introduction}
\label{sec:intro}

Wikipedia is een voorbeeld van de grootste online community van collaborerende mensen op het internet. Op Wikipedia zijn er veel users die edits maken op pagina's. Zij maken nieuwe pagina's, bewerken pagina's of controleren bewerkingen van andere users. Achter elke pagina schuilt ook een discussiepagina. Deze discussiepagina's dienen voor coordinatie, discussies en communicaties tussen \textit{Wikipedians}. In tegenstelling tot andere veel andere websites, zijn discussiepagina's op Wikipedia dus niet bedoelt voor entertainment, maar om tot een consensus te komen wat zal zorgen tot verbetering van Wikipedia pagina's \citep{laniado2011wikipedians}.  Echter, uit een vragenlijst bleek dat minder dan 20\% van deze Wikipedians vrouw is \citep{glott2010analysis}. Dit grote verschil tussen aantal mannen en aantal vrouwen binnen deze Wikipedians heeft veel aandacht gekregen van zowel de media als andere onderzoekers. Voornamelijk omdat Wikipedia geen technische gemeenschap is, waar men meer mannen zou verwachten. Gender onderzoeksliteratuur \citep{collier2012conflict} suggereert dat dit verschil kan komen door:
\begin{enumerate}
    \item Het hoge aantal conflicten in discussies
    \item Afkeer van kritische omgevingen
    \item Te kort aan zelfvertrouwen bij het aanpassen van andermans werk.
\end{enumerate}  
Er zijn al een aantal onderzoeken gedaan naar het analyseren van de inhoud van Wikipedia discussiepagina's \citep{viegas2007talk} en de structuur van de discussiepagina's van Wikipedia \citep{laniado2011wikipedians}, maar nog niet naar het verschil tussen mannenlijke en vrouwlijke verschillen op discussiepagina's van Wikipedia. In deze thesis wordt er gekeken naar het verschil in gebruik van de discussiepagina's van Wikipedia tussen mannelijke en vrouwlijke Wikipedians.



\subsection{Kritiek}
Hoewel Wikipedia bekent staat dat alle mensen iets kunnen 'bijdragen' aan de gemeenschap, is het vaak zo dat 'bijdragen' staat voor het verwijderen of aanpassen van andermans werk. 'Samenwerken' betekent dus niet altijd dat iedereen collectief nieuwe informatie produceert. Vaak is 'samenwerken' het feit dat mensen het werk van onbekende anderen lezen, corrigeren en soms ook verwijderen. Er is al eerder onderzocht dat als kinderen, jongens sneller kiezen voor competitieve spelletjes, terwijl meisjes sneller kiezen voor spelletjes zonder een winnaar \citep{campbell2013mind}. Een reden hiervan is dat mannen meer houden van competitie. Vrouwen reageren ook anders op kritiek dan mannen. Mannen reageren minder op positieve of negatieve feedback dan vrouwen \citep{roberts1994gender}. Vrouwen krijgen meer zelfvertrouwen van positieve feedback terwijl mannen relatief onaangetast blijven. Bij het ontvangen van negatieve kritiek wordt het zelfvertrouwen van vrouwen aanzienlijk verlaagd terwijl dit van mannen wederom onaangetast blijft \citep{roberts1994gender}. Vrouwen gaven ook aan 31\% banger te zijn voor negatieve houdingen ten opzichten van hun werk op Wikipedia \citep{collier2012conflict}. Een reden voor het lage aantal vrouwelijke editors op Wikipedia zou dus kunnen zijn dat mannen negatiever zijn dan vrouwen. 

\begin{description}
\item[Onderzoeksvraag 1] Is het taalgebruik van mannelijke gebruikers negatiever dan het taalgebruik van vrouwelijke gebruikers op discussiepagina's van Wikipedia?
\end{description}

\subsection{Conflict}
Behalve de kritieke sfeer tussen Wikipedia editors, is er ook een hoog niveau aan conflicten. Hoewel het lijkt dat er veel wordt samengewerkt op Wikipedia, staan de discussiepagina's vol met \textit{edit-warring}: een fenomeen waar editors proberen andere edits op te heffen \citep{cassell2011editwars}. Volgens Cassell moeten editors erop aandringen dan hun standpunt de enige geldige is. Eerder onderzoek heeft al aangetoond dat vrouwen sneller geneigd zijn om conflicten te vermijden dan man \citep{brewer2002gender}. Ook ervaren vrouwen sneller angst en verhoogde hartslag bij conflicterende situaties \citep{smith1998agency}. Eerder onderzoek toonde ook aan dat vrouwen onderhandelingen en conflict zelfs vermijden als het gaat om hoge kosten. Een vrouw is bereid ongeveer \$1300,- meer te betalen voor een auto om onderhandelingen te vermijden \citep{babcock2009women}. Omdat Wikipedia's discussiepagina's grotendeels zijn gebaseerd op het oplossen van conflicten, persoonlijke discussies en onderhandelingen om tot een consensus te komen, zouden de verschillen in gender om mee te doen aan deze activiteiten de impact van contributie kunnen be\"{i}nvloeden.
\begin{description}
\item[Onderzoeksvraag 2] Zijn er vaak discussies tussen mannelijke gebruikers en vrouwelijke gebruikers op discussiepagina's?
\end{description}





\subsection{Inhoud van thesis}
* Hier geef ik kort weer wat in elke sectie van mijn thesis staat *
