\section{Introduction}
\label{sec:intro}

Wikipedia is een voorbeeld van de grootste online community van collaborerende mensen op het internet. Op Wikipedia zijn er veel users die edits maken op pagina's. Zij maken nieuwe pagina's, bewerken pagina's of controleren bewerkingen van andere users. Achter elke pagina schuilt ook een discussiepagina. Deze discussiepagina's dienen voor coordinatie, discussies en communicaties tussen \textit{Wikipedians}. In tegenstelling tot andere veel andere websites, zijn discussiepagina's op Wikipedia dus niet bedoelt voor entertainment, maar om tot een consensus te komen wat zal zorgen tot verbetering van Wikipedia pagina's \citep{laniado2011wikipedians}.  Echter, uit een vragenlijst bleek dat minder dan 20\% van deze Wikipedians vrouw is \citep{glott2010analysis}. Dit grote verschil tussen aantal mannen en aantal vrouwen binnen deze Wikipedians heeft veel aandacht gekregen van zowel de media als andere onderzoekers. Voornamelijk omdat Wikipedia geen technische gemeenschap is, waar men meer mannen zou verwachten. Gender onderzoeksliteratuur \citep{collier2012conflict} suggereert dat dit verschil kan komen door:
\begin{enumerate}
    \item Het hoge aantal conflicten in discussies
    \item Afkeer van kritische omgevingen
    \item Te kort aan zelfvertrouwen bij het aanpassen van andermans werk.
\end{enumerate}  
Er zijn al een aantal onderzoeken gedaan naar het analyseren van de inhoud van Wikipedia discussiepagina's \citep{viegas2007talk} en de structuur van de discussiepagina's van Wikipedia \citep{laniado2011wikipedians}, maar nog niet naar het verschil tussen mannenlijke en vrouwlijke verschillen op discussiepagina's van Wikipedia. In deze thesis wordt er gekeken naar het verschil in gebruik van de discussiepagina's van Wikipedia tussen mannelijke en vrouwlijke Wikipedians.



\subsection{Onderzoeksvragen}
Om verschil tussen mannen en vrouwen op Wikipedia talkpages beter te bekijken, zijn er een aantal onderzoeksvagen opgesteld. 

\begin{description}
\item[Onderzoeksvraag 1] Is het taalgebruik van mannelijke gebruikers negatiever dan het taalgebruik van vrouwelijke gebruikers op discussiepagina's van Wikipedia?
\end{description}


\begin{description}
\item[Onderzoeksvraag 2] Zijn er vaak discussies tussen mannelijke gebruikers en vrouwelijke gebruikers op discussiepagina's?
\end{description}

Zoals al eerder genoemd, suggereerden Collier and Bear (2012) dat \'e\'en
van de redenen voor een tekort aan vrouwelijke Wikipedians zou kunnen liggen aan het hoge aantal conflicten in discussies. Daarom zal er een analyse komen aan de hand van de discussies die gevoerd worden op Wikipedia tussen mannen en vrouwen.




\subsection{Inhoud van thesis}
* Hier geef ik kort weer wat in elke sectie van mijn thesis staat *
